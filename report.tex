\documentclass{article}

\usepackage{tikz} 

\usepackage{amsmath} % advanced math

% margins of 1 inch:
\setlength{\topmargin}{-.5in}
\setlength{\textheight}{9.5in}
\setlength{\oddsidemargin}{0in}
\setlength{\textwidth}{6.5in}

\usepackage{hyperref}
\hypersetup{
    colorlinks=true,
    linkcolor=blue,
    filecolor=magenta,      
    urlcolor=cyan,
    pdftitle={Overleaf Example},
    pdfpagemode=FullScreen,
    }

\begin{document}

    % https://stackoverflow.com/a/3408428/1164295
    \begin{minipage}[h]{\textwidth}
        \title{2022 Future Computing Summer Internship Project:\\(Creating examples of livelock for the Structural Simulation Toolkit Discrete Event Simulator)}
        \author{Melissa Jost\footnote{melissakjost@gmail.com}\ , 
        AnotherFirst AnotherLast\footnote{anemail@domain.com}}
        \date{\today}
            \maketitle
        \begin{abstract}
            % This challenge problem was solved / progress was made
            The focus of this project was the current lack of examples looking into various distributed system
            design problems, and our ability to detect these issues.  In order to address this, our goal within this 
            repository was to take the issue of livelock, simulate it within SST, and come up with a set of metrics to 
            help us analyze other systems to detect these issues.  More specifically, we decided to simulate the 
            dining philosopher's problem, and use different iterations of this problem to analyze if livelock had occurred.  
            While coming up with the simulation for this problem was relatively straight-forward, there is still more work 
            to be done in regards to solidifying and expanding on the metrics we chose to define this problem within distributed systems. 
        \end{abstract}
    \end{minipage}

\ \\
% see https://en.wikipedia.org/wiki/George_H._Heilmeier#Heilmeier's_Catechism

%\maketitle

\section{Project Description} % what problem is being addressed? 

% The challenge addressed by this work is ...
The challenge that is being addressed by this work is that while we have countless examples of 
what issues existed within distributed systems, as well as various ways to avoid them , there is 
still a lack of work that exists with regards to quantifying metrics to detect these issues.  
In addition, by creating these simulations, we provide additional examples of livelock, which can be 
helpful to both those working to better understand these issues, as well as those trying to 
understand how to use SST as a whole.


\section{Motivation} % Why does this work matter? Who cares? If you're successful, what difference does it make?

The users of this work include two main groups of people: those working on developing distributed systems, as well 
as those interested in learning how to use SST.  For those working on developments for distributed systems, this is 
useful because if we are able to develop a truly useful set of metrics for detecting these various design problems, 
even if in simple simulations, it could potentially save a lot of time in finding them in real systems in the future.  
For those learning SST, these simulations help give more, simple examples of how to use the toolkit.

\section{Prior work} % what does this build on?

See \cite{texbook} for prior work in this area.

\section{How to do the thing}

% The software developed to respond to this challenge was run on (one laptop/a cluster of 100 nodes).
For the simulation, we are able to run it with one laptop, with a total of five philosophers.  This is how a 
majority of our analysis took place.  However, by including additional links and components, we could easily 
scale this simulation to model a bigger system.

The software is available on (https://github.com/lpsmodsim/2022HPCSummer-Livelock)

In order to run it, you simply need to run the Makefile.

\section{Models} % explanation of our SST model
Throughout the process of simulating the dining philosopher's problem, we were able to come up with a couple different instances of the 
problem, as well as note down certain parameters that seemed to make livelock more or less likely to occur.  When looking into what components 
made the most sense to model, we had different iterations: one that focused on modeling philosopher components alongside a central dining table 
component, as well as philosophers alongside individual chopstick components.  This allowed us to explore the ways in which different simulations 
would potentially affect the likelihood, as well as the ways in which livelock would occur.  From here, we set a list of parameters in our 
simulations to see what would induce livelock.  These parameters included:

\begin{itemize}
    \item thinkingDuration: The amount of time a philosopher spends thinking after placing down his chopsticks, whether that was after he finished eating, or while trying to allow another philosopher to eat
    \item waitingDuration: The frequency at which a philosopher checks the state of his hands, and sees whether or not he needs to place down his chopsticks to allow another philosopher to eat
    \item eatingDuration: The amount of time a philosopher spends holding both chopsticks when eating
    \item randomseed: This is a seed for our random number generator that randomizes our first pass at grabbing chopsticks
    \item id: This id allows the chopsticks or the dining table to be able to identify the philosopher
    \item livelockCheck: This indicates how many cycles we want to run our simulation for before exiting
    \item windowSize: This tells us how often we should check our system for potential livelock conditions
\end{itemize}

The most important parameters in this list are the thinkingDuration and the waitingDuration, because these set up 2 different sets of 
clocks within our simulation that run on different cycle lengths.  One cycle length is designated by thinkingDuration, and is where the 
philosopher sends requests to pick up chopsticks if necessary.  The other cycle length is designated by waitingDuration, which is what 
allows us to induce livelock.  Every time this clock cycle ticks, the philosopher checks the state of their hands, and if they’re only 
holding one chopstick, they put it back down to give someone else the opportunity to eat. This endless switching between looking for 
chopsticks in the thinking state, and placing them back down in the waiting state is what causes livelock to occur in the simulation.

In the model with a central dining table, we had 2 types of components: the dining philosophers and the dining table.  We created 5 
philosophers and 1 dining table in our python file, and gave each philosopher a link to the dining table.  By doing so, we held all 
our chopsticks in one central location and made sure there were no duplicate variables.  The philosophers would send message requests 
over their links if they needed to pick up a chopstick or put one back on the table.  The dining table received these events, and would 
let each philosopher know whether or not those chopsticks were actually available or not.  The information contained within each of those 
events is detailed in the "requests.h" header file.

In our other model, we replaced the central dining table component with individual chopsticks. Instead of each philosopher only having 
one link to the table, both philosophers and chopsticks had 2 links to keep track of.  Each philosopher had a link to both their left and 
right chopstick, and each chopstick had a link to their left and right philosopher.  In a similar fashion to the model described above, 
philosophers can take possession or place down their chopsticks by sending requests over their links.  The only difference here is that 
each chopstick only communicates with its adjacent philosophers as opposed to having one object managing the resources of the system.

One other model we had included, though not directly relevant to this issue of livelock, was a model that simulates deadlock.  The 
only difference between livelock and deadlock in the dining philosopher’s problem is that the philosophers won’t check to see if they 
need to place a chopstick down if they’re only being one.  Within SST, this is what our waiting clock does.  By removing the second clock, 
we force the philosophers into a deadlock, where each only holds one chopstick and no one ever eats.

When comparing both of these models, both of them produce very similar results, however, the model with individual chopsticks seemed 
to take longer to update their status.  We can assume this would be due to the fact that there are double the amount of links in the 
chopstick model.  We believe that the benefit that the chopstick model gives us is that from a logical point of view, it makes more 
sense to view the chopstick as its own component, as opposed to a dining table, which isn't’ something discussed as heavily in the 
definition of the dining philosopher’s problem.  However, the benefit that the dining table model gives us is that it means we already 
have a central node in place that can double as a way to collect information from all the philosophers.  In addition, since this model 
only requires each component to have one link, it’d be easier to extend this model to have more philosophers or chopsticks.

Lastly, an important thing to note about these models is that they were designed specifically with the idea in mind that there 
would be the same number of chopsticks and philosophers.  For example, the dining table keeps track of which philosophers would be 
allowed to access each chopstick. (ex. Chopstick “L5R1” is only accessible as Philosopher 1’s right chopstick, and Philosopher 5’s 
left chopstick).  The chopstick model also has a similar designation.  This would be important to keep in mind if you wanted to alter 
the simulation to have an uneven number of chopsticks to philosophers, where certain components weren’t predetermined to only be 
accessible to a specific subset of other components.  Generally speaking, the dining philosophers problem is designed to have an 
equal number of philosophers to chopsticks, but if you wanted to change the model in any way, this would be an important distinction.  

\section{Result} % conclusion/summary

% The result is ...
After we had established our models, we ran different sets of simulations on each set, observing when our philosophers actually 
experienced livelock.  The main thing that affected whether or not we experienced livelock was the changes we made to the timing 
parameters of the simulation.  Generally speaking, the more randomization you introduce into the timing, the less likely you are 
likely to experience livelock, with livelock being defined as the state in which none of the philosophers are ever able to enter 
their eating states.  In addition, you generally want your waitingDuration to be at most, equal to the thinkingDuration if you 
want to experience livelock.  In general, as long as each philosopher tries to grab the chopstick at the same time and puts them 
down at the same time, you are likely to enter a livelocked scenario.

In terms of the metrics we coined in order to detect livelock, our current findings is that they are heavily reliant on the 
specific problem at hand.  For example, much of the work in this area tends to agree that livelock is dependent on an existence 
of some infinite loop that contains changing states, as well as the lack of progress in the system.  In addition to this, many 
sources tend to agree that the definition of progress, as well as the definition of infinite is something that is left up to the 
programmer to decide, as it varies per problem.  We can define those criteria in the dining philosopher's problem by looking towards 
how much time one spends switching between thinking and being hungry, and how often one eats.  

Taking this information, we decided to define our metrics in terms of the progress we made in the system.  We let our simulation run 
for an extended period of time defined by our parameters, and at that point check in on the status of each of our states: thinking, 
hungry, and eating.  From here, we wait for another period of time that we designated as our "window". As this window closes, we once 
again take note of each of our states.  Once we have this data, we can first note whether or not we allowed the philosopher to eat at all.  
This tells us whether or not we made progress in our system.  After this, we can check the rate at which each philosopher switched from 
thinking to hungry.  We do this by measuring in our window the amount of switches that happened, and generating a percentage of time spent 
in each state based on what we expect.  In a livelock scenario, we would spend about half our time hungry, and half our time thinking, so 
we divide our window size by 2.  This gives us a fraction of the actual number of switches into a certain state over the expected number of 
switches into a certain state.  The higher each percentage is, the more likely it is to be livelocked, especially if there was no progress 
made in this window.  In true livelock, our hungry boolean would show us that no one ate, and that each of our percentages show that the 
philosopher spent all their time in either the hungry or thinking state.


\section{Future Work} % areas that still need more research
Overall, the lack of concrete research in regards to both livelock as a whole, as well as the specifics of how to quantify 
livelock issues limited this project.  Many papers had differing definitions on what qualified as livelock, and these various 
definitions were sometimes hard to translate into useful information for distributed systems, such as what would be considered 
an infinite loop, or what was meant by progress in the system.  To further understand this problem, it would probably be most 
helpful to find more real examples of livelock (ex. not the hallway problem), so that we can come up with metrics that are 
applicable to a wider set of systems.  While we can semi-confidently conclude that we have livelock in the simulations provided 
above, finding ways to generalize the process in which we found livelock would be useful  (ex. a more general definition of progress, 
or clearer examples of states that don't only apply to dining philosophers).

\begin{thebibliography}{9}
\bibitem{texbook}
Donald E. Knuth (1986) \emph{The \TeX{} Book}, Addison-Wesley Professional.

\bibitem{lamport94}
Leslie Lamport (1994) \emph{\LaTeX: a document preparation system}, Addison
Wesley, Massachusetts, 2nd ed.
\end{thebibliography}

\end{document}
